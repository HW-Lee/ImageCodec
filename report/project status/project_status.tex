%!TEX encoding = UTF-8 Unicode
\documentclass[12pt]{article} 
\usepackage[left=0.75in,top=0.7in,right=0.75in,bottom=0.3in]{geometry} % Document margins
\usepackage{CJK}
\usepackage{graphicx}
\usepackage{mathtools}
\usepackage{mathrsfs}
\usepackage{amssymb}
\usepackage{hyperref}
\usepackage{sidecap}
\usepackage{makecell}

\makeatletter
\renewenvironment{itemize}
{\list{$\bullet$}{\leftmargin\z@ \labelwidth\z@ \itemindent-\leftmargin
\let\makelabel\descriptionlabel}}
{\endlist}
\makeatother

\begin{CJK}{UTF8}{bsmi}
\title{\textbf{Project Status / Image Coding Contest }}
\author{\textbf{李豪韋 103061527} \\ \textbf{陳俐安 101061117}}
\date{}

\begin{document}
\vspace*{-60pt}
    {\let\newpage\relax\maketitle}

\section*{Status}
\vspace{-20pt}
\noindent\makebox[\linewidth]{\rule{\textwidth}{0.4pt}}
\vspace{5pt}

目前尚未實作,僅稍微規劃其結構。基本架構會採用transform coding \& predictive coding,先以無損壓縮開始研究,之後才會去研究quantization的部分。關於quantization有幾種選擇的方法:對於將256個數字縮到N個,無疑是開出N個區間然後將每一個值指定為對應區間的代表值,可以使用 1) k-means clustering, 2) 用machine learning的方法找出能讓predicted residual entropy最低的區間分法。目前還未知,但是k-means clustering的方法應該比較易於實作,不曉得老師有無任何建議?

目前找到的paper:
\href{http://sun.aei.polsl.pl/~rstaros/papers/s2006-spe-sfalic.pdf}{Simple Fast and Adaptive Lossless Image Compression Algorithm, Roman Starosolski}


\end{CJK}
\end{document}